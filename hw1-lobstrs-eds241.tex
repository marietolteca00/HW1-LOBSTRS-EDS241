% Options for packages loaded elsewhere
\PassOptionsToPackage{unicode}{hyperref}
\PassOptionsToPackage{hyphens}{url}
\documentclass[
]{article}
\usepackage{xcolor}
\usepackage[margin=1in]{geometry}
\usepackage{amsmath,amssymb}
\setcounter{secnumdepth}{-\maxdimen} % remove section numbering
\usepackage{iftex}
\ifPDFTeX
  \usepackage[T1]{fontenc}
  \usepackage[utf8]{inputenc}
  \usepackage{textcomp} % provide euro and other symbols
\else % if luatex or xetex
  \usepackage{unicode-math} % this also loads fontspec
  \defaultfontfeatures{Scale=MatchLowercase}
  \defaultfontfeatures[\rmfamily]{Ligatures=TeX,Scale=1}
\fi
\usepackage{lmodern}
\ifPDFTeX\else
  % xetex/luatex font selection
\fi
% Use upquote if available, for straight quotes in verbatim environments
\IfFileExists{upquote.sty}{\usepackage{upquote}}{}
\IfFileExists{microtype.sty}{% use microtype if available
  \usepackage[]{microtype}
  \UseMicrotypeSet[protrusion]{basicmath} % disable protrusion for tt fonts
}{}
\makeatletter
\@ifundefined{KOMAClassName}{% if non-KOMA class
  \IfFileExists{parskip.sty}{%
    \usepackage{parskip}
  }{% else
    \setlength{\parindent}{0pt}
    \setlength{\parskip}{6pt plus 2pt minus 1pt}}
}{% if KOMA class
  \KOMAoptions{parskip=half}}
\makeatother
\usepackage{color}
\usepackage{fancyvrb}
\newcommand{\VerbBar}{|}
\newcommand{\VERB}{\Verb[commandchars=\\\{\}]}
\DefineVerbatimEnvironment{Highlighting}{Verbatim}{commandchars=\\\{\}}
% Add ',fontsize=\small' for more characters per line
\usepackage{framed}
\definecolor{shadecolor}{RGB}{248,248,248}
\newenvironment{Shaded}{\begin{snugshade}}{\end{snugshade}}
\newcommand{\AlertTok}[1]{\textcolor[rgb]{0.94,0.16,0.16}{#1}}
\newcommand{\AnnotationTok}[1]{\textcolor[rgb]{0.56,0.35,0.01}{\textbf{\textit{#1}}}}
\newcommand{\AttributeTok}[1]{\textcolor[rgb]{0.13,0.29,0.53}{#1}}
\newcommand{\BaseNTok}[1]{\textcolor[rgb]{0.00,0.00,0.81}{#1}}
\newcommand{\BuiltInTok}[1]{#1}
\newcommand{\CharTok}[1]{\textcolor[rgb]{0.31,0.60,0.02}{#1}}
\newcommand{\CommentTok}[1]{\textcolor[rgb]{0.56,0.35,0.01}{\textit{#1}}}
\newcommand{\CommentVarTok}[1]{\textcolor[rgb]{0.56,0.35,0.01}{\textbf{\textit{#1}}}}
\newcommand{\ConstantTok}[1]{\textcolor[rgb]{0.56,0.35,0.01}{#1}}
\newcommand{\ControlFlowTok}[1]{\textcolor[rgb]{0.13,0.29,0.53}{\textbf{#1}}}
\newcommand{\DataTypeTok}[1]{\textcolor[rgb]{0.13,0.29,0.53}{#1}}
\newcommand{\DecValTok}[1]{\textcolor[rgb]{0.00,0.00,0.81}{#1}}
\newcommand{\DocumentationTok}[1]{\textcolor[rgb]{0.56,0.35,0.01}{\textbf{\textit{#1}}}}
\newcommand{\ErrorTok}[1]{\textcolor[rgb]{0.64,0.00,0.00}{\textbf{#1}}}
\newcommand{\ExtensionTok}[1]{#1}
\newcommand{\FloatTok}[1]{\textcolor[rgb]{0.00,0.00,0.81}{#1}}
\newcommand{\FunctionTok}[1]{\textcolor[rgb]{0.13,0.29,0.53}{\textbf{#1}}}
\newcommand{\ImportTok}[1]{#1}
\newcommand{\InformationTok}[1]{\textcolor[rgb]{0.56,0.35,0.01}{\textbf{\textit{#1}}}}
\newcommand{\KeywordTok}[1]{\textcolor[rgb]{0.13,0.29,0.53}{\textbf{#1}}}
\newcommand{\NormalTok}[1]{#1}
\newcommand{\OperatorTok}[1]{\textcolor[rgb]{0.81,0.36,0.00}{\textbf{#1}}}
\newcommand{\OtherTok}[1]{\textcolor[rgb]{0.56,0.35,0.01}{#1}}
\newcommand{\PreprocessorTok}[1]{\textcolor[rgb]{0.56,0.35,0.01}{\textit{#1}}}
\newcommand{\RegionMarkerTok}[1]{#1}
\newcommand{\SpecialCharTok}[1]{\textcolor[rgb]{0.81,0.36,0.00}{\textbf{#1}}}
\newcommand{\SpecialStringTok}[1]{\textcolor[rgb]{0.31,0.60,0.02}{#1}}
\newcommand{\StringTok}[1]{\textcolor[rgb]{0.31,0.60,0.02}{#1}}
\newcommand{\VariableTok}[1]{\textcolor[rgb]{0.00,0.00,0.00}{#1}}
\newcommand{\VerbatimStringTok}[1]{\textcolor[rgb]{0.31,0.60,0.02}{#1}}
\newcommand{\WarningTok}[1]{\textcolor[rgb]{0.56,0.35,0.01}{\textbf{\textit{#1}}}}
\usepackage{graphicx}
\makeatletter
\newsavebox\pandoc@box
\newcommand*\pandocbounded[1]{% scales image to fit in text height/width
  \sbox\pandoc@box{#1}%
  \Gscale@div\@tempa{\textheight}{\dimexpr\ht\pandoc@box+\dp\pandoc@box\relax}%
  \Gscale@div\@tempb{\linewidth}{\wd\pandoc@box}%
  \ifdim\@tempb\p@<\@tempa\p@\let\@tempa\@tempb\fi% select the smaller of both
  \ifdim\@tempa\p@<\p@\scalebox{\@tempa}{\usebox\pandoc@box}%
  \else\usebox{\pandoc@box}%
  \fi%
}
% Set default figure placement to htbp
\def\fps@figure{htbp}
\makeatother
\setlength{\emergencystretch}{3em} % prevent overfull lines
\providecommand{\tightlist}{%
  \setlength{\itemsep}{0pt}\setlength{\parskip}{0pt}}
\usepackage{bookmark}
\IfFileExists{xurl.sty}{\usepackage{xurl}}{} % add URL line breaks if available
\urlstyle{same}
\hypersetup{
  pdftitle={Assignment 1},
  pdfauthor={Marie Tolteca - EDS 241 / ESM 244 (Due: 1/17)},
  hidelinks,
  pdfcreator={LaTeX via pandoc}}

\title{Assignment 1}
\usepackage{etoolbox}
\makeatletter
\providecommand{\subtitle}[1]{% add subtitle to \maketitle
  \apptocmd{\@title}{\par {\large #1 \par}}{}{}
}
\makeatother
\subtitle{California Spiny Lobster (\emph{Panulirus Interruptus}):
Assessing the Impact of Marine Protected Areas (MPAs) at 5 Reef Sites in
Santa Barbara County}
\author{Marie Tolteca - EDS 241 / ESM 244 (\textbf{Due: 1/17})}
\date{1/17/26}

\begin{document}
\maketitle

\begin{center}\rule{0.5\linewidth}{0.5pt}\end{center}

\pandocbounded{\includegraphics[keepaspectratio]{figures/spiny2.jpg}}

\begin{center}\rule{0.5\linewidth}{0.5pt}\end{center}

\subsubsection{Assignment Instructions:}\label{assignment-instructions}

\begin{itemize}
\item
  Working with partners to troubleshoot code and concepts is encouraged!
  If you work with a partner, please list their name next to yours at
  the top of your assignment so Annie and I can easily see who
  collaborated.
\item
  All written responses must be written independently (\textbf{in your
  own words}).
\item
  Please follow the question prompts carefully and include only the
  information each question asks in your submitted responses.
\item
  Submit both your knitted document and the associated
  \texttt{RMarkdown} or \texttt{Quarto} file.
\item
  Your knitted presentation should meet the quality you'd submit to
  research colleagues or feel confident sharing publicly. Refer to the
  rubric for details about presentation standards.
\end{itemize}

\textbf{Assignment submission (Marie Tolteca):}
\_\_\_\_\_\_\_\_\_\_\_\_\_\_\_\_\_\_\_\_\_\_\_\_\_\_\_\_\_\_\_\_\_\_\_\_\_\_

\begin{center}\rule{0.5\linewidth}{0.5pt}\end{center}

\begin{Shaded}
\begin{Highlighting}[]
\FunctionTok{library}\NormalTok{(tidyverse)}
\FunctionTok{library}\NormalTok{(here)}
\FunctionTok{library}\NormalTok{(janitor)}
\FunctionTok{library}\NormalTok{(estimatr)  }
\FunctionTok{library}\NormalTok{(performance)}
\FunctionTok{library}\NormalTok{(jtools)}
\FunctionTok{library}\NormalTok{(gt)}
\FunctionTok{library}\NormalTok{(gtsummary)}
\FunctionTok{library}\NormalTok{(interactions) }
\FunctionTok{library}\NormalTok{(ggridges)}
\FunctionTok{library}\NormalTok{(beeswarm)}
\end{Highlighting}
\end{Shaded}

\begin{center}\rule{0.5\linewidth}{0.5pt}\end{center}

\paragraph{DATA SOURCE:}\label{data-source}

\begin{quote}
\href{https://doi.org/10.6073/pasta/a593a675d644fdefb736750b291579a0}{Reed
D. 2019. SBC LTER: Reef: Abundance, size and fishing effort for
California Spiny Lobster (Panulirus interruptus), ongoing since 2012.
Environmental Data Initiative.} Data accessed 11/17/2019.
\end{quote}

\begin{center}\rule{0.5\linewidth}{0.5pt}\end{center}

\subsubsection{\texorpdfstring{\textbf{Introduction}}{Introduction}}\label{introduction}

You're about to dive into some deep data collected from five reef sites
in Santa Barbara County, all about the abundance of California spiny
lobsters! 🦞 Data was gathered by divers annually from 2012 to 2018
across Naples, Mohawk, Isla Vista, Carpinteria, and Arroyo Quemado
reefs.

Why lobsters? Well, this sample provides an opportunity to evaluate the
impact of Marine Protected Areas (MPAs) established on January 1, 2012
(Reed, 2019). Of these five reefs, Naples, and Isla Vista are MPAs,
while the other three are not protected (non-MPAs). Comparing lobster
health between these protected and non-protected areas gives us the
chance to study how commercial and recreational fishing might impact
these ecosystems.

We will consider the MPA sites the \texttt{treatment} group and use
regression methods to explore whether protecting these reefs really
makes a difference compared to non-MPA sites (our control group). In
this assignment, we'll think deeply about which causal inference
assumptions hold up under the research design and identify where they
fall short.

Let's break it down step by step and see what the data reveals! 📊

\pandocbounded{\includegraphics[keepaspectratio]{figures/map-5reefs.png}}

\begin{center}\rule{0.5\linewidth}{0.5pt}\end{center}

\paragraph{Step 1: Anticipating potential sources of selection
bias}\label{step-1-anticipating-potential-sources-of-selection-bias}

\textbf{a.} Do the control sites (Arroyo Quemado, Carpenteria, and
Mohawk) provide a strong counterfactual for our treatment sites (Naples,
Isla Vista)? Write a paragraph making a case for why this comparison is
ceteris paribus or whether selection bias is likely (be specific!).

From the image above, I believe this may be selection bias. The reason
is because we have more information on the control sites with a total of
3 sites and only 2 control sites. The location of the control sites are
close in proximity whereas the control sites are more scattered.

\begin{center}\rule{0.5\linewidth}{0.5pt}\end{center}

\paragraph{Step 2: Read \& wrangle data}\label{step-2-read-wrangle-data}

\textbf{a.} Read in the raw data from the ``data'' folder named
\texttt{spiny\_abundance\_sb\_18.csv}. Name the data.frame
\texttt{rawdata}

\textbf{b.} Use the function \texttt{clean\_names()} from the
\texttt{janitor} package

\begin{Shaded}
\begin{Highlighting}[]
\CommentTok{\# HINT: check for coding of missing values (\textasciigrave{}na = "{-}99999"\textasciigrave{})}

\NormalTok{rawdata }\OtherTok{\textless{}{-}} \FunctionTok{read\_csv}\NormalTok{(}\FunctionTok{here}\NormalTok{(}\StringTok{\textquotesingle{}data\textquotesingle{}}\NormalTok{,}\StringTok{\textquotesingle{}spiny\_abundance\_sb\_18.csv\textquotesingle{}}\NormalTok{), }\AttributeTok{na =} \StringTok{\textquotesingle{}{-}99999\textquotesingle{}}\NormalTok{) }\SpecialCharTok{\%\textgreater{}\%} 
    \FunctionTok{clean\_names}\NormalTok{()}
\end{Highlighting}
\end{Shaded}

\textbf{c.} Create a new \texttt{df} named \texttt{tidyata}. Using the
variable \texttt{site} (reef location) create a new variable
\texttt{reef} as a \texttt{factor} and add the following labels in the
order listed (i.e., re-order the \texttt{levels}):

\begin{verbatim}
"Arroyo Quemado", "Carpenteria", "Mohawk", "Isla Vista",  "Naples"
\end{verbatim}

\section{FIX}\label{fix}

\begin{Shaded}
\begin{Highlighting}[]
\CommentTok{\# FIX}
\NormalTok{tidydata }\OtherTok{\textless{}{-}}\NormalTok{ rawdata }\SpecialCharTok{\%\textgreater{}\%} 
    \FunctionTok{mutate}\NormalTok{(}\AttributeTok{reef =} \FunctionTok{case\_when}\NormalTok{(}
\NormalTok{        site }\SpecialCharTok{==} \StringTok{"AQUE"} \SpecialCharTok{\textasciitilde{}} \StringTok{"Arroyo Quemado"}\NormalTok{,}
\NormalTok{        site }\SpecialCharTok{==} \StringTok{"CARP"} \SpecialCharTok{\textasciitilde{}} \StringTok{"Carpenteria"}\NormalTok{,}
\NormalTok{        site }\SpecialCharTok{==} \StringTok{"MOHK"} \SpecialCharTok{\textasciitilde{}} \StringTok{"Mohawk"}\NormalTok{,}
\NormalTok{        site }\SpecialCharTok{==} \StringTok{"IVEE"} \SpecialCharTok{\textasciitilde{}} \StringTok{"Isla Vista"}\NormalTok{,}
\NormalTok{        site }\SpecialCharTok{==} \StringTok{"NAPL"} \SpecialCharTok{\textasciitilde{}} \StringTok{"Naples"}\NormalTok{),}
        \AttributeTok{reef =} \FunctionTok{factor}\NormalTok{(reef,}
                      \AttributeTok{levels =} \FunctionTok{c}\NormalTok{(}\StringTok{"Arroyo Quemado"}\NormalTok{,}\StringTok{"Carpenteria"}\NormalTok{,}\StringTok{"Mohawk"}\NormalTok{, }\StringTok{"Isla Vista"}\NormalTok{,}\StringTok{"Naples"}\NormalTok{)))}
\end{Highlighting}
\end{Shaded}

Create new \texttt{df} named \texttt{spiny\_counts}

\begin{Shaded}
\begin{Highlighting}[]
\NormalTok{spiny\_counts }\OtherTok{\textless{}{-}}\NormalTok{ tidydata}
\end{Highlighting}
\end{Shaded}

\textbf{d.} Create a new variable \texttt{counts} to allow for an
analysis of lobster counts where the unit-level of observation is the
total number of observed lobsters per \texttt{site}, \texttt{year} and
\texttt{transect}.

\begin{itemize}
\tightlist
\item
  Create a variable \texttt{mean\_size} from the variable
  \texttt{size\_mm}
\item
  NOTE: The variable \texttt{counts} should have values which are
  integers (whole numbers).
\item
  Make sure to account for missing cases (\texttt{na})!
\end{itemize}

\textbf{e.} Create a new variable \texttt{mpa} with levels \texttt{MPA}
and \texttt{non\_MPA}. For our regression analysis create a numerical
variable \texttt{treat} where MPA sites are coded \texttt{1} and
non\_MPA sites are coded \texttt{0}

\begin{Shaded}
\begin{Highlighting}[]
\CommentTok{\#HINT(d): Use \textasciigrave{}group\_by()\textasciigrave{} \& \textasciigrave{}summarize()\textasciigrave{} to provide the total number of lobsters observed at each site{-}year{-}transect row{-}observation. }

\NormalTok{count }\OtherTok{\textless{}{-}}\NormalTok{ tidydata }\SpecialCharTok{\%\textgreater{}\%} 
    \FunctionTok{group\_by}\NormalTok{(year, site, transect) }\SpecialCharTok{\%\textgreater{}\%} 
    \FunctionTok{summarize}\NormalTok{(}\AttributeTok{num\_lobsters =} \FunctionTok{sum}\NormalTok{(count,}\AttributeTok{na.rm=}\ConstantTok{TRUE}\NormalTok{), }
              \AttributeTok{mean\_size =} \FunctionTok{mean}\NormalTok{(size\_mm,}\AttributeTok{na.rm=}\ConstantTok{TRUE}\NormalTok{), }
              \AttributeTok{.groups =} \StringTok{"drop"}\NormalTok{)}

\CommentTok{\#HINT(e): Use \textasciigrave{}case\_when()\textasciigrave{} to create the 3 new variable columns}

\NormalTok{spiny\_counts }\OtherTok{\textless{}{-}}\NormalTok{ count }\SpecialCharTok{\%\textgreater{}\%} 
    \FunctionTok{mutate}\NormalTok{(}
        \AttributeTok{mpa =} \FunctionTok{case\_when}\NormalTok{(}
\NormalTok{            site }\SpecialCharTok{\%in\%} \FunctionTok{c}\NormalTok{(}\StringTok{\textquotesingle{}IVEE\textquotesingle{}}\NormalTok{,}\StringTok{\textquotesingle{}NAPL\textquotesingle{}}\NormalTok{) }\SpecialCharTok{\textasciitilde{}} \StringTok{"MPA"}\NormalTok{,}
\NormalTok{            site }\SpecialCharTok{\%in\%} \FunctionTok{c}\NormalTok{(}\StringTok{\textquotesingle{}AQUE\textquotesingle{}}\NormalTok{, }\StringTok{\textquotesingle{}CARP\textquotesingle{}}\NormalTok{, }\StringTok{\textquotesingle{}MOHK\textquotesingle{}}\NormalTok{) }\SpecialCharTok{\textasciitilde{}} \StringTok{"non\_MPA"}\NormalTok{),}
        \AttributeTok{treat =} \FunctionTok{case\_when}\NormalTok{(}
\NormalTok{            site }\SpecialCharTok{\%in\%} \FunctionTok{c}\NormalTok{(}\StringTok{\textquotesingle{}IVEE\textquotesingle{}}\NormalTok{,}\StringTok{\textquotesingle{}NAPL\textquotesingle{}}\NormalTok{) }\SpecialCharTok{\textasciitilde{}} \DecValTok{1}\NormalTok{,}
\NormalTok{            site }\SpecialCharTok{\%in\%} \FunctionTok{c}\NormalTok{(}\StringTok{\textquotesingle{}AQUE\textquotesingle{}}\NormalTok{, }\StringTok{\textquotesingle{}CARP\textquotesingle{}}\NormalTok{, }\StringTok{\textquotesingle{}MOHK\textquotesingle{}}\NormalTok{)  }\SpecialCharTok{\textasciitilde{}} \DecValTok{0} 
\NormalTok{        ))}
\end{Highlighting}
\end{Shaded}

\begin{quote}
NOTE: This step is crucial to the analysis. Check with a friend or come
to TA/instructor office hours to make sure the counts are coded
correctly!
\end{quote}

\begin{center}\rule{0.5\linewidth}{0.5pt}\end{center}

\paragraph{Step 3: Explore \& visualize
data}\label{step-3-explore-visualize-data}

\textbf{a.} Take a look at the data! Get familiar with the data in each
\texttt{df} format (\texttt{tidydata}, \texttt{spiny\_counts})

\textbf{b.} We will focus on the variables \texttt{count},
\texttt{year}, \texttt{site}, and \texttt{treat}(\texttt{mpa}) to model
lobster abundance. Create the following 4 plots using a different method
each time from the 6 options provided. Add a layer (\texttt{geom}) to
each of the plots including informative descriptive statistics (you
choose; e.g., mean, median, SD, quartiles, range). Make sure each plot
dimension is clearly labeled (e.g., axes, groups).

\begin{itemize}
\tightlist
\item
  \href{https://r-charts.com/distribution/density-plot-group-ggplot2}{Density
  plot}
\item
  \href{https://r-charts.com/distribution/ggridges/}{Ridge plot}
\item
  \href{https://ggplot2.tidyverse.org/reference/geom_jitter.html}{Jitter
  plot}
\item
  \href{https://r-charts.com/distribution/violin-plot-group-ggplot2}{Violin
  plot}
\item
  \href{https://r-charts.com/distribution/histogram-density-ggplot2/}{Histogram}
\item
  \href{https://r-charts.com/distribution/beeswarm/}{Beeswarm}
\end{itemize}

Create plots displaying the distribution of lobster \textbf{counts}:

\begin{enumerate}
\def\labelenumi{\arabic{enumi})}
\tightlist
\item
  grouped by reef site\\
\item
  grouped by MPA status
\item
  grouped by year
\end{enumerate}

Create a plot of lobster \textbf{size} :

\begin{enumerate}
\def\labelenumi{\arabic{enumi})}
\setcounter{enumi}{3}
\tightlist
\item
  You choose the grouping variable(s)!
\end{enumerate}

\begin{Shaded}
\begin{Highlighting}[]
\NormalTok{spiny\_counts }\SpecialCharTok{\%\textgreater{}\%} 
  \FunctionTok{mutate}\NormalTok{(}\AttributeTok{site =} \FunctionTok{fct\_rev}\NormalTok{(site)) }\SpecialCharTok{\%\textgreater{}\%}  
  \FunctionTok{ggplot}\NormalTok{(}\FunctionTok{aes}\NormalTok{(}\AttributeTok{x =}\NormalTok{ num\_lobsters, }\AttributeTok{y =}\NormalTok{ site, }\AttributeTok{fill =}\NormalTok{ site)) }\SpecialCharTok{+}
  \FunctionTok{geom\_density\_ridges}\NormalTok{(}\AttributeTok{alpha =} \FloatTok{0.7}\NormalTok{) }\SpecialCharTok{+}
  \FunctionTok{scale\_fill\_viridis\_d}\NormalTok{(}\AttributeTok{option =} \StringTok{"plasma"}\NormalTok{) }\SpecialCharTok{+}
  \FunctionTok{geom\_vline}\NormalTok{(}\AttributeTok{xintercept =} \FunctionTok{mean}\NormalTok{(spiny\_counts}\SpecialCharTok{$}\NormalTok{num\_lobsters), }
             \AttributeTok{linetype =} \StringTok{"dashed"}\NormalTok{, }\AttributeTok{color =} \StringTok{"red"}\NormalTok{, }\AttributeTok{size =} \FloatTok{0.5}\NormalTok{) }\SpecialCharTok{+} 
  \FunctionTok{labs}\NormalTok{(}
    \AttributeTok{title =} \StringTok{"Distribution of Lobster Counts by Site"}\NormalTok{,}
    \AttributeTok{x =} \StringTok{"Number of Lobsters"}\NormalTok{,}
    \AttributeTok{y =} \StringTok{"Site"}
\NormalTok{  ) }\SpecialCharTok{+}
  \FunctionTok{theme\_light}\NormalTok{() }\SpecialCharTok{+}
  \FunctionTok{theme}\NormalTok{(}\AttributeTok{legend.position =} \StringTok{"none"}\NormalTok{)}
\end{Highlighting}
\end{Shaded}

\begin{Shaded}
\begin{Highlighting}[]
\CommentTok{\# 2}
\NormalTok{spiny\_counts }\SpecialCharTok{\%\textgreater{}\%} 
  \FunctionTok{ggplot}\NormalTok{(}\FunctionTok{aes}\NormalTok{(}\AttributeTok{x =}\NormalTok{ mpa, }\AttributeTok{y =}\NormalTok{ num\_lobsters)) }\SpecialCharTok{+}
  \FunctionTok{geom\_violin}\NormalTok{(}\AttributeTok{fill =} \StringTok{\textquotesingle{}dodgerblue\textquotesingle{}}\NormalTok{, }\AttributeTok{alpha =} \FloatTok{0.6}\NormalTok{) }\SpecialCharTok{+}  
  \FunctionTok{geom\_jitter}\NormalTok{(}\AttributeTok{width =} \FloatTok{0.2}\NormalTok{, }\AttributeTok{alpha =} \FloatTok{0.5}\NormalTok{, }\AttributeTok{size =} \FloatTok{1.5}\NormalTok{) }\SpecialCharTok{+}  
    \CommentTok{\# Displaying median of number of lobsters}
  \FunctionTok{geom\_hline}\NormalTok{(}\FunctionTok{aes}\NormalTok{(}\AttributeTok{yintercept =} \FunctionTok{mean}\NormalTok{(num\_lobsters)), }
             \AttributeTok{linetype =} \StringTok{"dashed"}\NormalTok{, }\AttributeTok{color =} \StringTok{"red"}\NormalTok{, }\AttributeTok{size =} \DecValTok{1}\NormalTok{) }\SpecialCharTok{+} 
  \FunctionTok{labs}\NormalTok{(}
    \AttributeTok{title =} \StringTok{"Lobster Counts by MPA Status"}\NormalTok{,}
    \AttributeTok{x =} \StringTok{"Marine Protected Area (MPA) Status"}\NormalTok{,}
    \AttributeTok{y =} \StringTok{"Number of Lobsters"}
\NormalTok{  ) }\SpecialCharTok{+}
    \FunctionTok{scale\_x\_discrete}\NormalTok{(}\AttributeTok{labels =} \FunctionTok{c}\NormalTok{(}\StringTok{"MPA"}\NormalTok{, }\StringTok{"Non{-}MPA"}\NormalTok{))}\SpecialCharTok{+}
  \FunctionTok{theme\_light}\NormalTok{()}
\end{Highlighting}
\end{Shaded}

\begin{Shaded}
\begin{Highlighting}[]
\CommentTok{\# 3}
\NormalTok{spiny\_counts }\SpecialCharTok{\%\textgreater{}\%} 
  \FunctionTok{ggplot}\NormalTok{(}\FunctionTok{aes}\NormalTok{(}\AttributeTok{x =} \FunctionTok{as.factor}\NormalTok{(year), }\AttributeTok{y =}\NormalTok{ num\_lobsters, }\AttributeTok{color =} \FunctionTok{as.factor}\NormalTok{(year))) }\SpecialCharTok{+}
\NormalTok{  ggbeeswarm}\SpecialCharTok{::}\FunctionTok{geom\_beeswarm}\NormalTok{(}\AttributeTok{alpha =} \FloatTok{0.6}\NormalTok{, }\AttributeTok{size =} \DecValTok{2}\NormalTok{) }\SpecialCharTok{+}
  \FunctionTok{scale\_color\_viridis\_d}\NormalTok{(}\AttributeTok{option =} \StringTok{"plasma"}\NormalTok{) }\SpecialCharTok{+}
  \FunctionTok{geom\_hline}\NormalTok{(}\AttributeTok{yintercept =} \FunctionTok{median}\NormalTok{(spiny\_counts}\SpecialCharTok{$}\NormalTok{num\_lobsters), }
             \AttributeTok{linetype =} \StringTok{"dashed"}\NormalTok{, }\AttributeTok{color =} \StringTok{"red"}\NormalTok{, }\AttributeTok{size =} \DecValTok{1}\NormalTok{) }\SpecialCharTok{+} 
  \FunctionTok{labs}\NormalTok{(}
    \AttributeTok{title =} \StringTok{"Lobster Count Distribution by Year"}\NormalTok{,}
    \AttributeTok{x =} \StringTok{"Year"}\NormalTok{,}
    \AttributeTok{y =} \StringTok{"Number of Lobsters"}
\NormalTok{  ) }\SpecialCharTok{+}
  \FunctionTok{theme\_light}\NormalTok{() }\SpecialCharTok{+}
  \FunctionTok{theme}\NormalTok{(}\AttributeTok{legend.position =} \StringTok{"none"}\NormalTok{)}
\end{Highlighting}
\end{Shaded}

\begin{Shaded}
\begin{Highlighting}[]
\NormalTok{spiny\_counts }\SpecialCharTok{\%\textgreater{}\%} 
  \FunctionTok{mutate}\NormalTok{(}\AttributeTok{site =} \FunctionTok{fct\_rev}\NormalTok{(site)) }\SpecialCharTok{\%\textgreater{}\%}  
  \FunctionTok{ggplot}\NormalTok{(}\FunctionTok{aes}\NormalTok{(}\AttributeTok{y =}\NormalTok{ site, }\AttributeTok{x =}\NormalTok{ mean\_size, }\AttributeTok{fill =}\NormalTok{ site)) }\SpecialCharTok{+}
  \FunctionTok{geom\_boxplot}\NormalTok{(}\AttributeTok{alpha =} \FloatTok{0.7}\NormalTok{, }\AttributeTok{outlier.shape =} \ConstantTok{NA}\NormalTok{) }\SpecialCharTok{+}
\NormalTok{  ggbeeswarm}\SpecialCharTok{::}\FunctionTok{geom\_beeswarm}\NormalTok{(}\AttributeTok{alpha =} \FloatTok{0.4}\NormalTok{, }\AttributeTok{size =} \FloatTok{1.5}\NormalTok{) }\SpecialCharTok{+}
  \FunctionTok{scale\_fill\_viridis\_d}\NormalTok{(}\AttributeTok{option =} \StringTok{"viridis"}\NormalTok{) }\SpecialCharTok{+}
  \FunctionTok{labs}\NormalTok{(}
    \AttributeTok{y =} \StringTok{\textquotesingle{}Site\textquotesingle{}}\NormalTok{,}
    \AttributeTok{x =} \StringTok{\textquotesingle{}Mean Size of Lobsters (mm)\textquotesingle{}}\NormalTok{,}
    \AttributeTok{title =} \StringTok{"Distribution of Average Lobster Size by Site"}
\NormalTok{  ) }\SpecialCharTok{+}
  \FunctionTok{theme\_light}\NormalTok{() }\SpecialCharTok{+}
  \FunctionTok{theme}\NormalTok{(}\AttributeTok{legend.position =} \StringTok{"none"}\NormalTok{)}
\end{Highlighting}
\end{Shaded}

\textbf{c.} Compare means of the outcome by treatment group. Using the
\texttt{tbl\_summary()} function from the package
\href{https://www.danieldsjoberg.com/gtsummary/articles/tbl_summary.html}{\texttt{gt\_summary}}

\begin{Shaded}
\begin{Highlighting}[]
\CommentTok{\# USE: gt\_summary::tbl\_summary()}
\NormalTok{spiny\_counts }\SpecialCharTok{\%\textgreater{}\%} 
  \FunctionTok{select}\NormalTok{(mpa, num\_lobsters) }\SpecialCharTok{\%\textgreater{}\%} 
  \FunctionTok{tbl\_summary}\NormalTok{(}
    \AttributeTok{by =}\NormalTok{ mpa,}
    \CommentTok{\# Getting Calcs}
    \AttributeTok{statistic =} \FunctionTok{all\_continuous}\NormalTok{() }\SpecialCharTok{\textasciitilde{}} \StringTok{"\{mean\} (\{sd\})"}\NormalTok{, }
    \CommentTok{\# Better Labeling}
    \AttributeTok{label =}\NormalTok{ num\_lobsters }\SpecialCharTok{\textasciitilde{}} \StringTok{"Number of Lobsters"} 
\NormalTok{  ) }\SpecialCharTok{\%\textgreater{}\%}
  \FunctionTok{modify\_header}\NormalTok{(}
    \CommentTok{\# }
    \AttributeTok{stat\_1 =} \StringTok{"**MPA**"}\NormalTok{,}
    \AttributeTok{stat\_2 =} \StringTok{"**Non{-}MPA**"}
\NormalTok{  ) }\SpecialCharTok{\%\textgreater{}\%}
  \FunctionTok{bold\_labels}\NormalTok{()}
\end{Highlighting}
\end{Shaded}

\begin{center}\rule{0.5\linewidth}{0.5pt}\end{center}

\paragraph{Step 4: OLS regression- building
intuition}\label{step-4-ols-regression--building-intuition}

\textbf{a.} Start with a simple OLS estimator of lobster counts
regressed on treatment. Use the function \texttt{summ()} from the
\href{https://jtools.jacob-long.com/}{\texttt{jtools}} package to print
the OLS output

\textbf{b.} Interpret the intercept \& predictor coefficients \emph{in
your own words}. Use full sentences and write your interpretation of the
regression results to be as clear as possible to a non-academic
audience.

\begin{verbatim}
- Intercept: On average, there are approximate 28% number of lobsters in MPA's.
- non-mpa: The model estimates a reduction of 5% number of lobsters in Non-MPA's.
\end{verbatim}

\begin{Shaded}
\begin{Highlighting}[]
\CommentTok{\# }\AlertTok{NOTE}\CommentTok{: We will not evaluate/interpret model fit in this assignment (e.g., R{-}square)}

\NormalTok{m1\_ols }\OtherTok{\textless{}{-}} \FunctionTok{lm}\NormalTok{(num\_lobsters }\SpecialCharTok{\textasciitilde{}}\NormalTok{ mpa, }
             \AttributeTok{data =}\NormalTok{ spiny\_counts)}

\FunctionTok{summ}\NormalTok{(m1\_ols, }\AttributeTok{model.fit =} \ConstantTok{FALSE}\NormalTok{) }
\end{Highlighting}
\end{Shaded}

\textbf{c.} Check the model assumptions using the \texttt{check\_model}
function from the \texttt{performance} package

\textbf{d.} Explain the results of the 4 diagnostic plots. Why are we
getting this result? 1. In the first plot, the dots do not fall within
the line, and they should be generally curved around zero. - Overall,
The reason we might be getting these results is because this test might
not be adequent for the data we have. Another general linear model can
be used and run \texttt{check\_model} functions.

\begin{Shaded}
\begin{Highlighting}[]
\FunctionTok{check\_model}\NormalTok{(m1\_ols,  }\AttributeTok{check =} \StringTok{"qq"}\NormalTok{ )}
\end{Highlighting}
\end{Shaded}

\begin{enumerate}
\def\labelenumi{\arabic{enumi}.}
\setcounter{enumi}{1}
\tightlist
\item
  The residuals fall outside of the normal curve, although it is
  generally close to the normal curve.
\end{enumerate}

\begin{Shaded}
\begin{Highlighting}[]
\FunctionTok{check\_model}\NormalTok{(m1\_ols, }\AttributeTok{check =} \StringTok{"normality"}\NormalTok{)}
\end{Highlighting}
\end{Shaded}

\begin{enumerate}
\def\labelenumi{\arabic{enumi}.}
\setcounter{enumi}{2}
\tightlist
\item
  The reference line is not flat and horizontal, it follows the shape
  with variance around it.
\end{enumerate}

\begin{Shaded}
\begin{Highlighting}[]
\FunctionTok{check\_model}\NormalTok{(m1\_ols, }\AttributeTok{check =} \StringTok{"homogeneity"}\NormalTok{)}
\end{Highlighting}
\end{Shaded}

\begin{enumerate}
\def\labelenumi{\arabic{enumi}.}
\setcounter{enumi}{3}
\tightlist
\item
  The model predicted data does not resemeble the spike of the observed
  data, however it does represent a normal distribution..
\end{enumerate}

\begin{Shaded}
\begin{Highlighting}[]
\FunctionTok{check\_model}\NormalTok{(m1\_ols, }\AttributeTok{check =} \StringTok{"pp\_check"}\NormalTok{)}
\end{Highlighting}
\end{Shaded}

\begin{center}\rule{0.5\linewidth}{0.5pt}\end{center}

\paragraph{Step 5: Fitting GLMs}\label{step-5-fitting-glms}

\textbf{a.} Estimate a Poisson regression model using the \texttt{glm()}
function

\textbf{b.} Interpret the predictor coefficient in your own words. Use
full sentences and write your interpretation of the results to be as
clear as possible to a non-academic audience. - Intercept: On average,
the model estimates our MPA lobster count to have 28 lobsters. -
non-mpa: The model estimate that there is a reduction of 19 lobster
counts in non-mpa locations.

\textbf{c.} Explain the statistical concept of dispersion and
overdispersion in the context of this model. - Dispersion describes how
much the data can vary. If the mean number of lobsters at a site is 5,
the variance is assumed to also be 5. - Overdispersion occurs when there
is more variability in the data than expected. In our model this can be
represent by how the each sites ecosystem can be different.

\textbf{d.} Compare results with previous model, explain change in the
significance of the treatment effect - The difference between OLS and
Poisson, is that poisson is used for modeling count outcomes and with
the link function of log, it restricts values to only be positive
outcomes. The OLS is used for continuous approximations to normal
outcomes, assumptions are more complicated since it does not account for
discrete and skewed data. In the summary the OLS SE and P Values were
much higher than the poisson model, meaning they are not significant.

\begin{Shaded}
\begin{Highlighting}[]
\CommentTok{\#HINT1: Incidence Ratio Rate (IRR): Exponentiation of beta returns coefficient which is interpreted as the \textquotesingle{}percent change\textquotesingle{} for a one unit increase in the predictor }

\CommentTok{\#HINT2: For the second glm() argument \textasciigrave{}family\textasciigrave{} use the following specification option \textasciigrave{}family = poisson(link = "log")\textasciigrave{}}

\NormalTok{m2\_pois }\OtherTok{\textless{}{-}}\FunctionTok{glm}\NormalTok{(num\_lobsters }\SpecialCharTok{\textasciitilde{}}\NormalTok{ mpa,}
                  \AttributeTok{family =} \FunctionTok{poisson}\NormalTok{(}\AttributeTok{link =} \StringTok{"log"}\NormalTok{),}\CommentTok{\# Using log for multiplicative}
                  \AttributeTok{data =}\NormalTok{ spiny\_counts}
\NormalTok{                  )}

\FunctionTok{summ}\NormalTok{(m2\_pois, }\AttributeTok{model.fit =} \ConstantTok{FALSE}\NormalTok{)}
\CommentTok{\#\textgreater{} exp(3.34) = 28.21}
\CommentTok{\#\textgreater{} (exp({-}0.21){-}1)*100 = {-}18.94}
\end{Highlighting}
\end{Shaded}

\textbf{e.} Check the model assumptions. Explain results. In the
assumption of poisson, Not all counts fit the Poisson curve, mean ≠
dispersion (variance),and high proportion of zeros. Our results from the
poisson model is that both the SE and pvalue are much smaller, meaning
they are significant.

\textbf{f.} Conduct tests for over-dispersion \& zero-inflation. Explain
results.

When the \texttt{check\_model} was ran for the \texttt{m2\_pois} the
figures are more accurate, although there may be another model that may
be a better fit for our count data. In the results for overdispersion
test we got a high ratio of 67.03. From the high ratio, we can tell that
we our data has significant overdispersion. In the results for
zero-inflation we have a zero predicted zeroes. Although, there are 27
observed zeros, this means that the model is underfitting zeros.

\begin{Shaded}
\begin{Highlighting}[]
\FunctionTok{check\_model}\NormalTok{(m2\_pois)}
\end{Highlighting}
\end{Shaded}

\begin{Shaded}
\begin{Highlighting}[]
\FunctionTok{check\_overdispersion}\NormalTok{(m2\_pois)}
\end{Highlighting}
\end{Shaded}

\begin{Shaded}
\begin{Highlighting}[]
\FunctionTok{check\_zeroinflation}\NormalTok{(m2\_pois)}
\end{Highlighting}
\end{Shaded}

\textbf{g.} Fit a negative binomial model using the function glm.nb()
from the package \texttt{MASS} and check model diagnostics

\textbf{h.} In 1-2 sentences explain rationale for fitting this GLM
model. A negative binomial was used to model overdispersed count data,
it allows to account for over or under dispersion. While it is more
flexible than a poisson glm, the estimated effect suggests about a 19\%
reduction in lobster abundance at non-MPA sites, accounting for
overdispersion increases uncertainty and not significant.

\textbf{i.} Interpret the treatment estimate result in your own words.
Compare with results from the previous model. Intercept: Our model
estimates an average of 3.34\% (28 count) number of lobsters to be found
in MPA locations. non-mpa: The model estimate that there is a reduction
of 19\% lobster counts in non-mpa locations. Comparing the results with
the poisson model, the negative binomial fit the \texttt{check\_model}
very accurate. The distribution of residuals, posterior predictive
check, misspecified dispersion and zero-inflation plots fall closer to
each other than in the poisson model. Taking a look into the
overdispersion, there is a ratio of 1.400 and a p-value of 0.064,
smaller values than our previous model. The zero-inflation show the
lower observed zeros and higher predicted zeros showcasing the moderate
fitting of zeros in outcome.

\begin{Shaded}
\begin{Highlighting}[]
\FunctionTok{library}\NormalTok{(MASS) }\DocumentationTok{\#\# }\AlertTok{NOTE}\DocumentationTok{: The \textasciigrave{}select()\textasciigrave{} function is masked. Use: \textasciigrave{}dplyr::select()\textasciigrave{} \#\#}
\end{Highlighting}
\end{Shaded}

\begin{Shaded}
\begin{Highlighting}[]
\CommentTok{\# }\AlertTok{NOTE}\CommentTok{: The \textasciigrave{}glm.nb()\textasciigrave{} function does not require a \textasciigrave{}family\textasciigrave{} argument}

\NormalTok{m3\_nb }\OtherTok{\textless{}{-}}\NormalTok{ spiny\_counts }\SpecialCharTok{\%\textgreater{}\%} 
\NormalTok{    dplyr}\SpecialCharTok{::}\FunctionTok{select}\NormalTok{(num\_lobsters,mpa) }\SpecialCharTok{\%\textgreater{}\%} 
    \FunctionTok{glm.nb}\NormalTok{(num\_lobsters }\SpecialCharTok{\textasciitilde{}}\NormalTok{ mpa,}
           \AttributeTok{data  =}\NormalTok{ .)}

\FunctionTok{summ}\NormalTok{(m3\_nb, }\AttributeTok{model.fit =} \ConstantTok{FALSE}\NormalTok{)}
\end{Highlighting}
\end{Shaded}

\begin{Shaded}
\begin{Highlighting}[]
\FunctionTok{check\_overdispersion}\NormalTok{(m3\_nb)}
\end{Highlighting}
\end{Shaded}

\begin{Shaded}
\begin{Highlighting}[]
\FunctionTok{check\_zeroinflation}\NormalTok{(m3\_nb)}
\end{Highlighting}
\end{Shaded}

\begin{Shaded}
\begin{Highlighting}[]
\FunctionTok{check\_predictions}\NormalTok{(m3\_nb)}
\end{Highlighting}
\end{Shaded}

\begin{Shaded}
\begin{Highlighting}[]
\FunctionTok{check\_model}\NormalTok{(m3\_nb)}
\end{Highlighting}
\end{Shaded}

\begin{center}\rule{0.5\linewidth}{0.5pt}\end{center}

\paragraph{Step 6: Compare models}\label{step-6-compare-models}

\textbf{a.} Use the \texttt{export\_summ()} function from the
\texttt{jtools} package to look at the three regression models you fit
side-by-side.

\textbf{c.} Write a short paragraph comparing the results. Is the
treatment effect \texttt{robust} or stable across the model
specifications. The treatment effect is consistent across models,
indicating fewer lobsters in non-MPA locations. However, the statistical
significance of this effect is not robust to model choice. The Poisson
model finds a highly significant negative effect, but diagnostic checks
indicate overdispersion and excess zeros, which violate Poisson
assumptions. When this extra variability is accounted for in the
negative binomial model, SE increase and the treatment effect is no
longer statistically significant. This suggests that while the estimated
effect size is stable, the strength of the evidence for a treatment
effect depends on the modeling assumptions.

\begin{Shaded}
\begin{Highlighting}[]
\FunctionTok{export\_summs}\NormalTok{(}\CommentTok{\# ADD MODELS}
\NormalTok{    m1\_ols, m2\_pois, m3\_nb,}
             \AttributeTok{model.names =} \FunctionTok{c}\NormalTok{(}\StringTok{"OLS"}\NormalTok{,}\StringTok{"Poisson"}\NormalTok{, }\StringTok{"NB"}\NormalTok{),}
             \AttributeTok{statistics =} \StringTok{"none"}\NormalTok{)}
\end{Highlighting}
\end{Shaded}

\begin{center}\rule{0.5\linewidth}{0.5pt}\end{center}

\paragraph{Step 7: Building intuition - fixed
effects}\label{step-7-building-intuition---fixed-effects}

\textbf{a.} Create new \texttt{df} with the \texttt{year} variable
converted to a factor

\textbf{b.} Run the following negative binomial model using
\texttt{glm.nb()}

\begin{itemize}
\tightlist
\item
  Add fixed effects for \texttt{year} (i.e., dummy coefficients)
\item
  Include an interaction term between variables \texttt{treat} \&
  \texttt{year} (\texttt{treat*year})
\end{itemize}

\textbf{c.} Take a look at the regression output. Each coefficient
provides a comparison or the difference in means for a specific
sub-group in the data. Informally, describe the what the model has
estimated at a conceptual level (NOTE: you do not have to interpret
coefficients individually) - This model estimates differences in lobster
counts between treated and untreated sites while controlling year to
year. This interaction term being year, allows for variation and impact
in treatment areas over time.

\textbf{d.} Explain why the main effect for treatment is negative? *Does
this result make sense? The main effect for treatment is negative is
because the first implementations of being treated could have just
started. This means most of the area is barely being protected from the
public, new plants could have been plants and have not yet gone into
affect. This results does make sense since the first two years were
rocking but then it starts to improve over time. Even by the year 2013,
there is reduction of about 1.37 compared to the previous year.

\begin{Shaded}
\begin{Highlighting}[]
\NormalTok{ff\_counts }\OtherTok{\textless{}{-}}\NormalTok{ spiny\_counts }\SpecialCharTok{\%\textgreater{}\%} 
    \FunctionTok{mutate}\NormalTok{(}\AttributeTok{year=}\FunctionTok{as\_factor}\NormalTok{(year))}
    
\NormalTok{m5\_fixedeffs }\OtherTok{\textless{}{-}} \FunctionTok{glm.nb}\NormalTok{(}
\NormalTok{    num\_lobsters }\SpecialCharTok{\textasciitilde{}} 
\NormalTok{        treat }\SpecialCharTok{+}
\NormalTok{        year }\SpecialCharTok{+}
\NormalTok{        treat}\SpecialCharTok{*}\NormalTok{year,}
    \AttributeTok{data =}\NormalTok{ ff\_counts)}

\FunctionTok{summ}\NormalTok{(m5\_fixedeffs, }\AttributeTok{model.fit =} \ConstantTok{FALSE}\NormalTok{)}
\end{Highlighting}
\end{Shaded}

\textbf{e.} Look at the model predictions: Use the
\texttt{interact\_plot()} function from package \texttt{interactions} to
plot mean predictions by year and treatment status.

\textbf{f.} Re-evaluate your responses (c) and (b) above. In both plot,
treated areas were lower than untreated locations. However, in both
plots, it is notable that treated/ MPA locations have more abundance of
lobsters compared to non-mpa locations.

\begin{Shaded}
\begin{Highlighting}[]
\NormalTok{p1 }\OtherTok{\textless{}{-}} \FunctionTok{interact\_plot}\NormalTok{(m5\_fixedeffs, }\AttributeTok{pred =}\NormalTok{ year, }\AttributeTok{modx =}\NormalTok{ treat,}
              \AttributeTok{outcome.scale =} \StringTok{"link"}\NormalTok{) }\SpecialCharTok{+} \CommentTok{\# }\AlertTok{NOTE}\CommentTok{: y{-}axis on log{-}scale}
    \FunctionTok{labs}\NormalTok{(}\AttributeTok{title =} \StringTok{"Link Space"}\NormalTok{,}
         \AttributeTok{x =} \StringTok{\textquotesingle{}Year\textquotesingle{}}\NormalTok{,}
         \AttributeTok{y =} \StringTok{\textquotesingle{}Lobster Count\textquotesingle{}}\NormalTok{)}

\CommentTok{\# HINT: Change \textasciigrave{}outcome.scale\textasciigrave{} to "response" to convert y{-}axis scale to counts}

\NormalTok{p2 }\OtherTok{\textless{}{-}} \FunctionTok{interact\_plot}\NormalTok{(m5\_fixedeffs, }\AttributeTok{pred =}\NormalTok{ year, }\AttributeTok{modx =}\NormalTok{ treat,}
              \AttributeTok{outcome.scale =} \StringTok{"response"}\NormalTok{) }\SpecialCharTok{+} \CommentTok{\# }\AlertTok{NOTE}\CommentTok{: y{-}axis on log{-}scale}
    \FunctionTok{labs}\NormalTok{(}\AttributeTok{title =} \StringTok{"Count Space"}\NormalTok{,}
         \AttributeTok{x =} \StringTok{\textquotesingle{}Year\textquotesingle{}}\NormalTok{,}
         \AttributeTok{y =} \StringTok{\textquotesingle{}Lobster Count\textquotesingle{}}\NormalTok{,}
         \AttributeTok{modx =} \StringTok{\textquotesingle{}Treatment\textquotesingle{}}\NormalTok{)}

\NormalTok{p1 }\SpecialCharTok{+}\NormalTok{ p2}
\end{Highlighting}
\end{Shaded}

\textbf{g.} Using \texttt{ggplot()} create a plot in same style as the
previous \texttt{interaction\ plot}, but displaying the original scale
of the outcome variable (lobster counts). This type of plot is commonly
used to show how the treatment effect changes across discrete time
points (i.e., panel data).

The plot should have\ldots{} - \texttt{year} on the x-axis -
\texttt{counts} on the y-axis - \texttt{mpa} as the grouping variable

\begin{Shaded}
\begin{Highlighting}[]
\CommentTok{\# Hint 1: Group counts by \textasciigrave{}year\textasciigrave{} and \textasciigrave{}mpa\textasciigrave{} and calculate the \textasciigrave{}mean\_count\textasciigrave{}}
\CommentTok{\# Hint 2: Convert variable \textasciigrave{}year\textasciigrave{} to a factor}

\NormalTok{plot\_counts }\OtherTok{\textless{}{-}}\NormalTok{ ff\_counts }\SpecialCharTok{\%\textgreater{}\%} 
    \FunctionTok{mutate}\NormalTok{(}\AttributeTok{year=}\FunctionTok{as\_factor}\NormalTok{(year)) }\SpecialCharTok{\%\textgreater{}\%} 
    \FunctionTok{group\_by}\NormalTok{(year, mpa) }\SpecialCharTok{\%\textgreater{}\%} 
    \FunctionTok{summarize}\NormalTok{(}\AttributeTok{mean\_count =} \FunctionTok{mean}\NormalTok{(num\_lobsters, }\AttributeTok{na.rm =} \ConstantTok{TRUE}\NormalTok{))}

\CommentTok{\# plot\_counts \%\textgreater{}\% ggplot() ...}
\NormalTok{plot\_counts }\SpecialCharTok{\%\textgreater{}\%} \FunctionTok{ggplot}\NormalTok{(}
    \FunctionTok{aes}\NormalTok{(}\AttributeTok{x=}\NormalTok{ year, }\AttributeTok{y =}\NormalTok{ mean\_count, }\AttributeTok{group =}\NormalTok{ mpa, }\AttributeTok{color =}\NormalTok{ mpa))}\SpecialCharTok{+}
    \FunctionTok{geom\_line}\NormalTok{() }\SpecialCharTok{+} 
    \FunctionTok{geom\_point}\NormalTok{() }\SpecialCharTok{+}
    \FunctionTok{scale\_color\_manual}\NormalTok{(}\AttributeTok{values =} 
        \FunctionTok{c}\NormalTok{(}\StringTok{"MPA"} \OtherTok{=} \StringTok{"dodgerblue4"}\NormalTok{,}
        \StringTok{"non\_MPA"} \OtherTok{=} \StringTok{"lightblue"}\NormalTok{),}
        \AttributeTok{labels =} \FunctionTok{c}\NormalTok{(}\StringTok{"MPA"}\NormalTok{, }\StringTok{"Non{-}MPA"}\NormalTok{)}
\NormalTok{        ) }\SpecialCharTok{+}
    \FunctionTok{labs}\NormalTok{(}\AttributeTok{x =} \StringTok{"Year"}\NormalTok{,}
         \AttributeTok{y =} \StringTok{"Mean Count (Lobsters)"}\NormalTok{,}
         \AttributeTok{title =} \StringTok{"Mean Lobster Count by Year and MPA Status"}\NormalTok{,}
         \AttributeTok{color  =}  \StringTok{"MPA"}\NormalTok{) }\SpecialCharTok{+}
    \FunctionTok{theme\_light}\NormalTok{()}
\end{Highlighting}
\end{Shaded}

\begin{center}\rule{0.5\linewidth}{0.5pt}\end{center}

\paragraph{Step 8: Reconsider causal identification
assumptions}\label{step-8-reconsider-causal-identification-assumptions}

\begin{enumerate}
\def\labelenumi{\alph{enumi}.}
\tightlist
\item
  Discuss whether you think \texttt{spillover\ effects} are likely in
  this research context (see Glossary of terms;
  \url{https://docs.google.com/document/d/1RIudsVcYhWGpqC-Uftk9UTz3PIq6stVyEpT44EPNgpE/edit?usp=sharing})
\end{enumerate}

\begin{itemize}
\tightlist
\item
  I do not think that there was a spillover effect in this study, since
  it would be hard to improve a non-mpa location without any help or
  vise versa, damage a MPA site. With an established MPA, it makes sense
  that the abundance of lobster count increases. If the number did not
  increase, there would have to be other accommodations.
\end{itemize}

\begin{enumerate}
\def\labelenumi{\alph{enumi}.}
\setcounter{enumi}{1}
\tightlist
\item
  Explain why spillover is an issue for the identification of causal
  effects
\end{enumerate}

\begin{itemize}
\tightlist
\item
  Spillover is an issue for the identification of causal effect because
  it can impact overall measurements of the study causing
  mis-interpretations and biased results.
\end{itemize}

\begin{enumerate}
\def\labelenumi{\alph{enumi}.}
\setcounter{enumi}{2}
\tightlist
\item
  How does spillover relate to impact in this research setting?
\end{enumerate}

\begin{itemize}
\tightlist
\item
  The spillover related to the impact in this research by the number of
  lobsters found in treated and untreated areas.
\end{itemize}

\begin{enumerate}
\def\labelenumi{\alph{enumi}.}
\setcounter{enumi}{3}
\item
  Discuss the following causal inference assumptions in the context of
  the MPA treatment effect estimator. Evaluate if each of the assumption
  are reasonable:

  \begin{enumerate}
  \def\labelenumii{\arabic{enumii})}
  \tightlist
  \item
    SUTVA: Stable Unit Treatment Value assumption
  \end{enumerate}

  \begin{itemize}
  \tightlist
  \item
    Causal inference assumption: No interference: One unit's treatment
    does not affect another unit's outcome (MPA/Treat). No hidden
    variation: The treatment is implemented consistently for all units
    (i.e., all units receive the same treatment)
  \item
    In the case of the study, I do not believe that there were any
    assumption violated. The overall goal of the study was to measure
    reef health and lobster abundance in MPA and non MPA areas.
  \end{itemize}

  \begin{enumerate}
  \def\labelenumii{\arabic{enumii})}
  \setcounter{enumii}{1}
  \tightlist
  \item
    Excludability assumption The excludability assumption requires that
    the intervention influences the outcome only through the proposed
    mechanisms, such as leader efforts and community coordination, and
    not through other unintended pathways.
  \end{enumerate}

  \begin{itemize}
  \tightlist
  \item
    An intervention could be a fisherman could have gone to an MPA site
    without anyone noticing. The researcher may have made an assumption
    that no boundaries were crossed.
  \end{itemize}
\end{enumerate}

\begin{center}\rule{0.5\linewidth}{0.5pt}\end{center}

\section{EXTRA CREDIT}\label{extra-credit}

\begin{quote}
Use the recent lobster abundance data with observations collected up
until 2024 (\texttt{extracredit\_sblobstrs24.csv}) to run an analysis
evaluating the effect of MPA status on lobster counts using the same
focal variables.
\end{quote}

\begin{enumerate}
\def\labelenumi{\alph{enumi}.}
\tightlist
\item
  Create a new script for the analysis on the updated data
\item
  Run at least 3 regression models \& assess model diagnostics
\item
  Compare and contrast results with the analysis from the 2012-2018 data
  sample (\textasciitilde{} 2 paragraphs)
\end{enumerate}

\begin{center}\rule{0.5\linewidth}{0.5pt}\end{center}

\pandocbounded{\includegraphics[keepaspectratio]{figures/spiny1.png}}

\end{document}
